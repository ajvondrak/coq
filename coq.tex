\documentclass{beamer}

\usetheme{CambridgeUS}

\setbeamertemplate{theorems}[normal font]
\setbeamertemplate{navigation symbols}{}

\newcommand{\exampletext}[1]{%
  \begin{beamercolorbox}{block title example}#1\end{beamercolorbox}%
}
\renewcommand<>{\fbox}[1]{\alt#2{\beameroriginal{\fbox}{#1}}{#1}}
\newcommand{\gray}[1]{\textcolor{darkgray}{#1}}

\usepackage{listings,lstlangcoq,bold-extra}
\lstset{basicstyle=\ttfamily,language=Coq,showstringspaces=false}

\usepackage{graphics}

\title{What's in a Proof?}
\author{Alex Vondrak}
\institute{Cal Poly Pomona}
\titlegraphic{\includegraphics[height=3cm]{four-lights.jpg}}

\begin{document}

\maketitle

\begin{frame}{An Anecdote}
  % Classmates: whinge about proving things in such minute detail that you'd
  % even need to prove (something like) 2 + 2 = 4.
  % Dr. Rich's reaction: "um, no".
  \begin{center}
    \includegraphics[width=10cm]{math-is-hard.png}
  \end{center}
\end{frame}

\begin{frame}[fragile]{The Reason}
  \begin{block}{}
    \begin{lstlisting}[language={},gobble=6]
      Theorem a: 2 + 2 = 4.
    \end{lstlisting}
  \end{block}
  \begin{block}{}<2>
    \begin{lstlisting}[language={},gobble=6]
      Proof.
        trivial.
      Qed.
    \end{lstlisting}
  \end{block}
\end{frame}

\begin{frame}[fragile]{Coq}
  \begin{center}
    \includegraphics[scale=0.4]{coq.png}
  \end{center}
  \begin{itemize}
    \item An \alert{interactive theorem prover} started in 1984
    \item Provides a formal language and environment for mathematical
          definitions, algorithms, theorems, and machine-checked proofs
    \item Language based on a derivative of the \alert{calculus of
          constructions} (\alert{CoC})
  \end{itemize}
  \begin{example}
    \begin{lstlisting}[gobble=6,escapechar=|]
      Theorem two_and_two_make_four: 2 + 2 = 4.
      Proof.
        |\alt<2>{\alert{auto 1.}}{trivial.}|
      Qed.
    \end{lstlisting}
  \end{example}
\end{frame}

\begin{frame}[fragile]{Proof Automation}
  \begin{block}{Rough Algorithm}
    \begin{lstlisting}[language={},
                       morekeywords={if,then,foreach,in,try},
                       mathescape=true,
                       escapechar=|,
                       gobble=6]
      auto n =
        if |\normalfont no more subgoals| then
          |\exampletext{success}|
        if n == 0 then
          |\alert{failure}|
        foreach |\normalfont term| in |\fbox{$hypotheses \cup hints$}|:
          try
            |\fbox{apply \text{\normalfont term}.}|
            foreach |\text{\normalfont subgoal generated}|:
              auto (n - 1) |\normalfont on that subgoal|
    \end{lstlisting}
  \end{block}
\end{frame}

\begin{frame}[fragile,t]{\texttt{apply} term\texttt{.}}
  \begin{itemize}
    \item Tries to \alert{unify} the goal with the conclusion of the
          \alert{type} of ``term''
    \item Returns \alert{subgoals}---premises of the type of ``term''
  \end{itemize}
  \begin{onlyenv}<1>
    \begin{example}[At the Coq Top-Level]
      \begin{lstlisting}[gobble=8,escapechar=|]
        |\gray{Coq <}| Example ex: (1=2 -> 2=1) -> (2=1 -> 1=2) -> 1=2.
        |\gray{1 subgoal}|

          |\gray{============================}|
           |\gray{(1 = 2 -> 2 = 1) -> (2 = 1 -> 1 = 2) -> 1 = 2}|

        |\gray{ex <}|
      \end{lstlisting}
    \end{example}
  \end{onlyenv}
  \begin{onlyenv}<2>
    \begin{example}[At the Coq Top-Level]
      \begin{lstlisting}[gobble=8,escapechar=|]
          |\gray{============================}|
           |\gray{(1 = 2 -> 2 = 1) -> (2 = 1 -> 1 = 2) -> 1 = 2}|

        |\gray{ex <}| intros.
        |\gray{1 subgoal}|

          |\gray{H : 1 = 2 -> 2 = 1}|
          |\gray{H0 : 2 = 1 -> 1 = 2}|
          |\gray{============================}|
           |\gray{1 = 2}|
      \end{lstlisting}
    \end{example}
  \end{onlyenv}
  \begin{onlyenv}<3>
    \begin{example}[At the Coq Top-Level]
      \begin{lstlisting}[gobble=8,escapechar=|]
          |\gray{H : 1 = 2 -> 2 = 1}|
          |\gray{H0 : 2 = 1 -> 1 = 2}|
          |\gray{============================}|
           |\gray{1 = 2}|

        |\gray{ex <}| apply H.
        |\gray{Toplevel input, characters 6-7:}|
        |\gray{> apply H.}|
        |\gray{>}|      |\gray{\^{}}|
        |\gray{Error: Impossible to unify "2 = 1" with "1 = 2".}|
      \end{lstlisting}
    \end{example}
  \end{onlyenv}
  \begin{onlyenv}<4>
    \begin{example}[At the Coq Top-Level]
      \begin{lstlisting}[gobble=8,escapechar=|]
        |\gray{ex <}| apply H0.
        |\gray{1 subgoal}|

          |\gray{H : 1 = 2 -> 2 = 1}|
          |\gray{H0 : 2 = 1 -> 1 = 2}|
          |\gray{============================}|
           |\gray{2 = 1}|

        |\gray{ex <}|
      \end{lstlisting}
    \end{example}
  \end{onlyenv}
\end{frame}

\begin{frame}[fragile]{Hints and Hypotheses}
  \begin{lstlisting}[gobble=4,escapechar=|]
    |\gray{Coq <}| Theorem two_and_two_make_four: 2 + 2 = 4.
    |\gray{1 subgoal}|

      |\gray{============================}|
       |\gray{2 + 2 = 4}|

    |\gray{two\_and\_two\_make\_four <}| Print Hint.
  \end{lstlisting}
  \small
  \begin{lstlisting}[language={},gobble=4]
    Applicable Hints :
    [...]
    In the database core:
      apply mult_n_O(0) apply mult_n_Sm(0) apply plus_n_O(0)
      apply eq_refl(0) apply plus_n_Sm(0)
      apply eq_add_S ; trivial(1) apply eq_sym ; trivial(1)
      apply f_equal (A:=nat)(1) apply f_equal2 mult(2)
      apply f_equal2 (A1:=nat) (A2:=nat)(2)
    [...]
  \end{lstlisting}
\end{frame}

\begin{frame}[fragile]{Hints and Hypotheses}{Which Hint?}
  \begin{lstlisting}[gobble=4,escapechar=|]
    |\gray{two\_and\_two\_make\_four <}| Proof.

    |\gray{two\_and\_two\_make\_four <}| info trivial.
     |\gray{==}| apply eq_refl.

    |\gray{Proof completed.}|

    |\gray{two\_and\_two\_make\_four}| < Qed.
    |\gray{info trivial.}|

    |\gray{two\_and\_two\_make\_four is defined}|

    |\gray{Coq <}|
  \end{lstlisting}
\end{frame}

\begin{frame}[fragile]{Equality}
  \begin{definition}
    \begin{lstlisting}[gobble=6]
      Inductive eq (A:Type) (x:A) : A -> Prop :=
        eq_refl : eq A x x.
    \end{lstlisting}
  \end{definition}
  \begin{itemize}
    \item \lstinline|eq_refl| is a \alert{constructor} of a proposition
    \begin{itemize}
      \item Given evidence that \lstinline|eq A x x|, \ldots
      \item \ldots \lstinline|eq_refl| allows us to conclude the proposition is
            true
    \end{itemize}
    \item \lstinline|eq_refl| ``is a proof of'' \lstinline|eq A x x|
    \item It's the \alert{only} way to prove something of type
          \lstinline|eq|---thus, this defines the smallest reflexive relation
  \end{itemize}
\end{frame}

\begin{frame}[fragile]{How Does That Help?}
  As it turns out, Coq tricks us a little\ldots
  \begin{block}{}
    \begin{lstlisting}[gobble=6,escapechar=|]
      |\gray{Coq <}| Set Printing All.

      |\gray{Coq <}| Print two_and_two_make_four.
      two_and_two_make_four =
      @eq_refl nat (S (S (S (S O))))
           : @eq nat (plus (S (S O)) (S (S O)))
                     (S (S (S (S O))))
    \end{lstlisting}
  \end{block}
  \begin{alertblock}{Note}
    The \lstinline|@|-sign has to do with making implicit arguments explicit
    for a particular function application (\lstinline|2 + 2 = 4| leaves the
    type \lstinline|nat| implicit)
  \end{alertblock}
\end{frame}

\begin{frame}[fragile]{Peano Arithmetic}
  \begin{definition}
    \begin{lstlisting}[gobble=6]
      Inductive nat : Set :=
        | O : nat
        | S : nat -> nat.
    \end{lstlisting}
  \end{definition}
  \begin{definition}
    \begin{lstlisting}[gobble=6]
      Fixpoint plus (n m:nat) : nat :=
        match n with
        | O => m
        | S p => S (p + m)
        end

      where "n + m" := (plus n m) : nat_scope.
    \end{lstlisting}
  \end{definition}
\end{frame}

\begin{frame}[fragile]{Predicative Calculus of (Co)Inductive Constructions}
  \begin{alertblock}{Problem}
    How can we \lstinline|apply eq_refl.| to
    \begin{lstlisting}[gobble=6]
      @eq nat (plus (S (S O)) (S (S O)))
              (S (S (S (S O))))
    \end{lstlisting}
    when we need evidence of the form \lstinline|eq A x x|?
  \end{alertblock}
  \begin{exampleblock}{Answer}
    The underlying formal language of Coq defines semantics for definitions,
    types, and (moreover) \alert{evaluation}:
    \begin{itemize}
      \item $\delta$-reduction
      \item $\iota$-reduction
      \item $\beta$-reduction
      \item $\zeta$-reduction (similar to $\delta$)
    \end{itemize}
  \end{exampleblock}
\end{frame}

\begin{frame}[fragile]{$\delta$-Reduction}
  Replaces definition names with their values
  \begin{example}
    \begin{lstlisting}[gobble=6,escapechar=$]
      @eq nat (plus (S (S O)) (S (S O)))
              (S (S (S (S O))))

      $\gray{Coq <}$ cbv delta.

      @eq nat
          ($\alert{(}$fix plus (n m : nat) : nat :=
              match n return nat with
              | O => m
              | S p => S (plus p m)
              end$\alert{)}$ (S (S O)) (S (S O)))
          (S (S (S (S O))))
    \end{lstlisting}
  \end{example}
\end{frame}

\begin{frame}[fragile,t]{$\iota$-Reduction}
  A specific conversion rule; for our purposes, expands \lstinline|Fixpoint|
  definitions
  \begin{example}
    \begin{onlyenv}<1>
      \begin{lstlisting}[gobble=8,escapechar=$]
        @eq nat ($\alert{(}$fix plus (n m : nat) : nat :=
                    match n return nat with
                    | O => m
                    | S p => S (plus p m)
                    end$\alert{)}$ (S (S O)) (S (S O)))
                (S (S (S (S O))))

        $\gray{Coq <}$ cbv iota.
      \end{lstlisting}
    \end{onlyenv}
    \begin{onlyenv}<2>
      \begin{lstlisting}[gobble=8,escapechar=$]
        @eq nat ($\alert{(}$fun n m : nat =>
                  match n return nat with
                  | O => m
                  | S p =>
                      S ($\alert{(}$fix plus (n0 m0 : nat) : nat :=
                            match n0 return nat with
                            | O => m0
                            | S p0 => S (plus p0 m0)
                            end$\alert{)}$ p m)
                  end$\alert{)}$ (S (S O)) (S (S O)))
                (S (S (S (S O))))
      \end{lstlisting}
    \end{onlyenv}
  \end{example}
\end{frame}

\begin{frame}[fragile]{$\beta$-Reduction}
  Applies \lstinline|fun|ctions to their arguments
  \begin{example}
    \begin{onlyenv}<1>
      \begin{lstlisting}[gobble=8,escapechar=$]
        @eq nat ($\alert{(}$fun n m : nat =>
                  match n return nat with
                  | O => m
                  | S p =>
                      S ($\alert{(}$fix plus (n0 m0 : nat) : nat :=
                            match n0 return nat with
                            | O => m0
                            | S p0 => S (plus p0 m0)
                            end$\alert{)}$ p m)
                  end$\alert{)}$ (S (S O)) (S (S O)))
                (S (S (S (S O))))
      \end{lstlisting}
    \end{onlyenv}
    \begin{onlyenv}<2>
      \begin{lstlisting}[gobble=8,escapechar=$]
        $\gray{Coq <}$ cbv beta.

        @eq nat (match $\alert{S (S O)}$ return nat with
                | O => $\alert{S (S O)}$
                | S p =>
                    S ((fix plus (n m : nat) : nat :=
                          match n return nat with
                          | O => m
                          | S p0 => S (plus p0 m)
                          end) p $\alert{(S (S O))}$)
                end)
                (S (S (S (S O))))
      \end{lstlisting}
    \end{onlyenv}
  \end{example}
\end{frame}

\begin{frame}[fragile,t]{$\iota$-Reduction}
  Here, $\iota$ resolves the \lstinline|match|
  \begin{example}
    \begin{onlyenv}<1>
      \begin{lstlisting}[gobble=8,escapechar=$]
        @eq nat (match $\alert{S (S O)}$ return nat with
                | O => $\alert{S (S O)}$
                | S p =>
                    S ((fix plus (n m : nat) : nat :=
                          match n return nat with
                          | O => m
                          | S p0 => S (plus p0 m)
                          end) p $\alert{(S (S O))}$)
                end)
                (S (S (S (S O))))
      \end{lstlisting}
    \end{onlyenv}
    \begin{onlyenv}<2>
      \begin{lstlisting}[gobble=8,escapechar=$]
        $\gray{Coq <}$ cbv iota.

        @eq nat ($\alert{(}$fun p : nat =>
                   S ($\alert{(}$fix plus (n m : nat) : nat :=
                        match n return nat with
                        | O => m
                        | S p0 => S (plus p0 m)
                        end$\alert{)}$ p (S (S O)))$\alert{)}$ $\alert{(S O)}$)
                (S (S (S (S O))))
      \end{lstlisting}
    \end{onlyenv}
  \end{example}
\end{frame}

\begin{frame}[fragile,t]{$\beta$-Reduction}
  And $\beta$ once again applies the \lstinline|fun|ction\ldots
  \begin{example}
    \begin{onlyenv}<1>
      \begin{lstlisting}[gobble=8,escapechar=$]
        @eq nat ($\alert{(}$fun p : nat =>
                   S ($\alert{(}$fix plus (n m : nat) : nat :=
                        match n return nat with
                        | O => m
                        | S p0 => S (plus p0 m)
                        end$\alert{)}$ p (S (S O)))$\alert{)}$ $\alert{(S O)}$)
                (S (S (S (S O))))
      \end{lstlisting}
    \end{onlyenv}
    \begin{onlyenv}<2>
      \begin{lstlisting}[gobble=8,escapechar=$]
        $\gray{Coq <}$ cbv beta.

        @eq nat (S ($\alert{(}$fix plus (n m : nat) : nat :=
                      match n return nat with
                      | O => m
                      | S p => S (plus p m)
                      end$\alert{)}$ (S O) (S (S O))))
                (S (S (S (S O))))
      \end{lstlisting}
    \end{onlyenv}
  \end{example}
\end{frame}

\begin{frame}[fragile,t]{Eventually\ldots}
  $\iota$ resolves yet another \lstinline|match|
  \begin{example}
    \begin{onlyenv}<1>
      \begin{lstlisting}[gobble=8,escapechar=$]
        @eq nat (S (S (match $\alert{O}$ return nat with
                       | $\alert{O}$ => S (S O)
                       | S p =>
                         S ((fix plus (n m : nat) : nat :=
                              match n return nat with
                              | O => m
                              | S p0 => S (plus p0 m)
                              end) p (S (S O)))
                       end)))
                (S (S (S (S O))))
      \end{lstlisting}
    \end{onlyenv}
    \begin{onlyenv}<2>
      \begin{lstlisting}[gobble=8,escapechar=$]
        $\gray{Coq <}$ cbv iota.

        @eq nat (S (S (S (S O))))
                (S (S (S (S O))))
      \end{lstlisting}
    \end{onlyenv}
  \end{example}
\end{frame}

\begin{frame}[fragile]{Finally}
  So, what we really mean by
  \begin{lstlisting}[gobble=4,frame=single]
    Theorem two_and_two_make_four: 2 + 2 = 4.
    Proof. trivial. Qed.
  \end{lstlisting}
  is
  \begin{lstlisting}[gobble=4,frame=single]
    Theorem there_are_four_lights: 2 + 2 = 4.
    Proof.
      cbv delta.
      cbv iota. cbv beta. cbv iota. cbv beta.
      cbv iota. cbv beta. cbv iota. cbv beta.
      cbv iota. cbv beta. cbv iota.
      apply eq_refl.
    Qed.
  \end{lstlisting}
  none of which is really necessary!
\end{frame}

\begin{frame}{Curry-Howard Correspondence}
  \begin{block}{In A Nutshell}
    \begin{align*}
      \text{propositions} &\quad\equiv\quad \text{types} \\
      \text{proofs}       &\quad\equiv\quad \text{programs}
    \end{align*}
    (Coq can even export proofs as programs in other languages!)
  \end{block}
  Strong future in automating proofs:
  \begin{itemize}
    \item Not just results that humans can't achieve on their own
    \begin{itemize}
      \item Though this \alert{has} happened (e.g., Robbins' conjecture)
      \item Ask a math student about this sometime!
    \end{itemize}
    \item End of the day: proof \alert{assistants} (e.g., 4-color theorem,
    Compcert)
  \end{itemize}
\end{frame}

\end{document}
